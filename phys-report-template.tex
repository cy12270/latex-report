\documentclass[a4paper,12pt]{article}

% Packages
% \usepackage[utf8]{inputenc}
\usepackage{amsmath, amssymb}
\usepackage{graphicx}
\usepackage{float}
\usepackage{hyperref}
\usepackage[margin=1in]{geometry}
\usepackage{siunitx} % For SI units
\usepackage{caption}
\usepackage{subcaption}
\usepackage{booktabs} % For tables
\usepackage{mhchem} % For chemical formulas

\usepackage{newtxtext} % Use Times-like font

\usepackage{multirow}
\usepackage[yyyymmdd]{datetime}

\tolerance=1
\emergencystretch=\maxdimen
\hyphenpenalty=10000
\hbadness=10000

\setlength{\parindent}{0pt}
\setlength{\parskip}{1em}

\renewcommand{\dateseparator}{-}

% Bibliography package with APA style
\usepackage[style=apa, backend=biber]{biblatex}
\addbibresource{references.bib} % Reference file

% Title page information
\title{Thermal Conductivity of Solids at Low Temperatures}
\author{Your Name}
\date{\today}

\begin{document}
\maketitle
\begin{abstract}
    This report presents the methodology and results of an experiment conducted to measure the thermal conductivity of various solid materials at low temperatures. The results are analyzed, compared to literature values, and discussed in the context of thermal conduction principles.
\end{abstract}
\newpage

% Sections of the Lab Report

\section{Introduction and theory}
\label{sec:introduction}
The objective of this experiment is to measure the thermal conductivity of solid samples at low temperatures using the axial flow method. According to Fourier's law, the rate of heat transfer is proportional to the temperature gradient across the material, which allows for the determination of thermal conductivity, \( k \). The experiment aims to investigate the temperature dependence of thermal conductivity for different materials, primarily nickel and glass, and to compare the experimental results to known literature values. \autocite{example1}. \par
Thermal conductivity describes a material's ability to conduct heat. It quantifies the rate at which thermal energy is transferred through a material due to a temperature gradient. A high thermal conductivity indicates that a material readily allows heat to flow through it, while a low thermal conductivity signifies that the material is a poor conductor of heat and acts as an insulator. Mathematically, this relationship is expressed by:
\begin{align}
\frac{ \mathrm{ d } Q }{ \mathrm{ d } t } = -kA\frac{ \mathrm{ d } T }{ \mathrm{ d } x }
% Q = -k \nabla T
\end{align}
where $\vec{q}$ is the heat flux density (rate of heat transfer per unit area), $k$ is the thermal conductivity of the material, and $\nabla T$ is the temperature gradient. The negative sign indicates that heat flows in the direction of decreasing temperature.

\section{Apparatus and experimental methods}
\label{sec:procedure}
The experimental setup for measuring thermal conductivity involves using a Janis SuperVariTemp cryostat, which allows for precise temperature control from 80\si{\kelvin} to room temperature. The sample, either a nickel rod or a glass plate, is mounted onto a heater within the cryostat, ensuring that the heat flow is uniaxial through the sample to minimize radial losses. Two thermocouples are attached to either side of the sample to measure the temperature difference ($\Delta T$), while the power input to the heater is recorded to calculate the heat flux. The experiment is conducted by cooling the system to thermal equilibrium, varying the temperature, and recording the necessary data at multiple points to analyze the thermal conductivity as a function of temperature.

\section{Results and discussion}
\label{sec:results}
Based on the experimental data, the thermal conductivity of the nickel sample decreases from approximately 94.7\si{\watt\per\meter\per\kelvin} at 80\si{\kelvin} to 54.4\si{\watt\per\meter\per\kelvin} at 293\si{\kelvin}. This decreasing trend is consistent with the theoretical predictions on the $k$ values of nickel, which suggests that $k$ decreases with rising temperature. However, the deviation between the experimental data of $k$ and the literature values lies in the magnitude of $k$. As shown by figure, the values of $k$ obtained in both samples are roughly 40 - 55\% lower than literature at similar temperature. This discrepancy is possibly due to the composition of the samples, as the impurities in the sample can reduce its $k$ values by promoting electron scattering \autocite{chen2021effects} and hence the heat transfer becomes less efficient.\par
Another possible source of the deviation is the error in experimental setup and procedure. In this experiment, power is supplied to one end, and the temperature difference is measured along the sample. If heat is lost from the surface of the sample to the surrounding environment via convection and radiation, the actual heat flowing through the measured section of the sample is less than the total power input. These heat losses generally increase with increasing temperature difference between the sample and the surroundings, becoming less significant at higher temperatures \autocite{corsan1992axial}. Using the total power input in the calculation while ignoring heat losses would lead to an underestimation of thermal conductivity, and this underestimation would be less pronounced at higher temperatures, explaining the observed decreasing trend of percentage difference.\par
While the experimental data of both nickel samples deviates from the literature values, the difference between the $k$ obtained from these two datasets is subtle, despite their different diameter. This similarity indicates that the diameter difference within this range is not as dominant as the material composition or general experimental limitations.\par
When comparing the experimental $k$ values from the mesurements using a glass plate with the literature data for quartz glass, a consistent trend that $k$ increases as the temperature rises is observed in both datasets. Similar to the previous measurements for nickel samples, the experimental $k$ values are substantially lower than the literature values for quartz glass across the entire measured temperature range. The percentage differences in table confirm this, showing the experimental values are roughly 23\% to 44\% lower than the literature values at comparable temperatures.\par
The most significant factor leading to this discrepancy is likely the difference in the composition of glass. The literature data we used is specifically for quartz glass (\ce{SiO2}), which does not contain additional components present in other types of glass \autocite{onoda1999introduction}. The addition of these components introduces various effects depending on the type and amount of the additive. For example, when alkali oxides are added to the silica (\ce{SiO2}) network, they break Si-O-Si bonds and provide more scattering centres for phonons \autocite{article}, leading to a decrease in $k$ compared to quartz glass.

\section{Discussion}
\label{sec:discussion}
Analyze the results and discuss their implications. Address any discrepancies or uncertainties and propose possible sources of error. Compare the experimental results with theoretical predictions if applicable.

\section{Conclusion}
\label{sec:conclusion}
Summarize the key findings of the experiment. Discuss whether the objectives of the experiment were achieved and suggest improvements for future experiments.

\section{References}
\label{sec:references}
The references are formatted according to the APA citation style. All citations in the text should use the \texttt{\textbackslash autocite} or \texttt{\textbackslash cite} commands.

\printbibliography % Automatically generates the bibliography

% Appendix (if needed)
\newpage
\appendix
\section{Appendix: tables and figures}
\label{sec:appendix}
Include any additional material such as raw data, derivations, or supplementary information.

\begin{table}[H]
    \centering
    \begin{tabular}{|c|c|c|}
        \hline
        Variable & Measured Value & Uncertainty \\ 
        \hline
        $x$ & $10.5$ & $\pm0.2$ \\ 
        $y$ & $15.3$ & $\pm0.1$ \\ 
        \hline
    \end{tabular}
    \caption{Sample data table.}
    \label{tab:example_table}
\end{table}

\end{document}
