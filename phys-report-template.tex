\documentclass[a4paper,12pt]{article}

% Packages
\usepackage[utf8]{inputenc}
\usepackage{amsmath, amssymb}
\usepackage{graphicx}
\usepackage{float}
\usepackage{hyperref}
\usepackage[margin=1in]{geometry}
\usepackage{siunitx} % For SI units
\usepackage{caption}
\usepackage{subcaption}
\usepackage{booktabs} % For tables
\usepackage{mhchem} % For chemical formulas

\usepackage{multirow}
\usepackage[yyyymmdd]{datetime}

\tolerance=1
\emergencystretch=\maxdimen
\hyphenpenalty=10000
\hbadness=10000

\renewcommand{\dateseparator}{-}

% Bibliography package with APA style
\usepackage[style=apa, backend=biber]{biblatex}
\addbibresource{references.bib} % Reference file

% Title page information
\title{Physics Lab Report}
\author{Your Name}
\date{\today}

\begin{document}
\maketitle
\newpage

% Title Page
% \maketitle
% \begin{center}
%     \rule{\textwidth}{0.5mm} \[1em]
%     \textbf{Experiment Title:} \[0.5em]
%     \emph{Your Experiment Name Here} \[1em]
%     \textbf{Date of Experiment:} \[0.5em]
%     \emph{Date Here} \[1em]
%     \rule{\textwidth}{0.5mm}
% \end{center}

% \vfill
% \begin{center}
%     \textbf{Abstract}
% \end{center}
% \begin{abstract}
%     Write a brief summary of the purpose, methods, results, and conclusions of the experiment. Aim for 150-200 words.
% \end{abstract}

% \newpage

% Table of Contents
% \tableofcontents
% \newpage

% Sections of the Lab Report

\section{Introduction}
\label{sec:introduction}
Provide background information, theoretical concepts, and the purpose of the experiment. Include relevant equations and define all variables. Cite sources as needed, e.g., \autocite{example1}.

\section{Experimental Procedure}
\label{sec:procedure}
Describe the experimental setup and procedure. Include diagrams or images if necessary. Be clear and concise, providing enough detail for someone to replicate the experiment.

\section{Results}
\label{sec:results}
Present the raw data and processed results. Use tables and graphs where appropriate. Ensure all figures and tables are labeled and referenced in the text.

% \begin{figure}[H]
%     \centering
%     \includegraphics[width=0.7\textwidth]{example_graph.png} % Replace with your file
%     \caption{Example graph showing experimental results.}
%     \label{fig:example_graph}
% \end{figure}

\begin{table}[H]
    \centering
    \begin{tabular}{|c|c|c|}
        \hline
        Variable & Measured Value & Uncertainty \\ 
        \hline
        $x$ & $10.5$ & $\pm0.2$ \\ 
        $y$ & $15.3$ & $\pm0.1$ \\ 
        \hline
    \end{tabular}
    \caption{Sample data table.}
    \label{tab:example_table}
\end{table}

\section{Discussion}
\label{sec:discussion}
Analyze the results and discuss their implications. Address any discrepancies or uncertainties and propose possible sources of error. Compare the experimental results with theoretical predictions if applicable.

\section{Conclusion}
\label{sec:conclusion}
Summarize the key findings of the experiment. Discuss whether the objectives of the experiment were achieved and suggest improvements for future experiments.

\section{References}
\label{sec:references}
The references are formatted according to the APA citation style. All citations in the text should use the \texttt{\textbackslash autocite} or \texttt{\textbackslash cite} commands.

\printbibliography % Automatically generates the bibliography

% Appendix (if needed)
\appendix
\section{Appendix}
\label{sec:appendix}
Include any additional material such as raw data, derivations, or supplementary information.

\end{document}
